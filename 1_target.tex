\chapter{目標}
接續專案二, 對雙輪車進行設計改良,  以提升行進與對戰效能, 採用CAD 進行場景與多輪車零組件設計後, 轉入足球場景中以鍵盤 arrow keys 與 wzas 等按鍵進行控制, 對陣雙方每組將有四名輪車球員, 且每兩人在同一台電腦上操作。
球賽計分系統必須採 .ttm 格式建立 (0~99), 使能通用於各類場景計數之用, 並可擴增至三位數計分,除了採用 LED 顯示計分外, 另外以建立機械轉盤傳動計分系統 。

\section{規格}

足球規格 (ball): 白色, 直徑 0.1m, 重量 0.5kg \\
足球場地 (field): 長 4m x 寬 2.5m \\
球門規格 (goal[0] and goal[1]: 長 0.6m, 高 0.3m, 寬 0.1m \\
球員尺寸範圍 (player[0]-player[8]: 長寬高各 0.2m, 重量 5kg \\

\section{規則}

一開始球放在正中央,雙方各球員分別用控制程式控制,開始賽局,時間內得最多分者獲勝。 \\
遊戲規則如下: \\
球進球框得一分。 \\
在規定時間內得最多分者獲勝。 \\
有一方進球後,球會重新從中間開始。 \\
