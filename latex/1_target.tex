\chapter{目標}
接續專案二, 對雙輪車進行設計改良,  以提升行進與對戰效能, 採用CAD 進行場景與多輪車零組件設計後, 轉入足球場景中以鍵盤 arrow keys 與 wzas 等按鍵進行控制, 對陣雙方每組將有四名輪車球員, 且每兩人在同一台電腦上操作。
球賽計分系統必須採 .ttm 格式建立 (0~99), 使能通用於各類場景計數之用, 並可擴增至三位數計分,除了採用 LED 顯示計分外, 另外以建立機械轉盤傳動計分系統 。

\section{規格}
\begin{tabular}{p{4cm}p{6cm}}
  \textbf{足球規格 (ball):} & 白色, 直徑 0.1m, 重量 0.5kg \\
  \textbf{足球場地 (field):} & 長 4m x 寬 2.5m \\
  \textbf{球門規格 (goal[0] 和 goal[1]):} & 長 0.6m, 高 0.3m, 寬 0.1m \\
  \textbf{球員尺寸範圍 (player[0]-player[8]):} & 長寬高各 0.2m, 重量 5kg \\
\end{tabular}

\section{規則}
球放在正中央,雙方球員各別控制,時間內得最多分者獲勝。 \\
遊戲規則如下: \\
\begin{enumerate}[label=\textbullet]
\item 球進球框得一分 \\
\item 在規定時間內得最多分者獲勝 \\
\item 有一方進球後,球會重新從中間開始 \\
\end{enumerate}
