\chapter{目標}
接續專案二, 對雙輪車進行設計改良, 以提升行進與對戰效能, 採用 CAD 進行場景與多輪車零組件設計後, 轉入足球場景中以鍵盤 arrow keys 與 wzas 等按鍵進行控制, 對陣雙方每組將有四名輪車球員, 且每兩人在同一台電腦上操作。
球賽計分系統必須採 .ttm 格式建立 (0~99), 使能通用於各類場景計數之用, 並可擴增至三位數計分,除了採用 LED 顯示計分外, 另外以建立機械轉盤傳動計分系統。

\section{規格}
\begin{tabular}{p{8cm}p{cm}}
  \textbf{足球規格 :} & 白色, 直徑 0.1m, 重量 0.5kg \\
  \textbf{足球場地 :} & 長 4m x 寬 2.5m \\
  \textbf{球門規格 :} & 長 0.6m, 高 0.3m, 寬 0.1m \\
  \textbf{球員尺寸範圍 :} & 長寬高各 0.2m, 重量 5kg \\
\end{tabular}

\section{規則}
球放在正中央,雙方球員各別控制,時間內得最多分者獲勝。 \\
遊戲規則如下:
\begin{tabular}{p{8cm}p{cm}}
  \textbf{球進球門得一分}\\
  \textbf{在規定時間內得最多分者獲勝}\\
  \textbf{有一方進球後,球會重新從中間開始}\\
\end{tabular}
