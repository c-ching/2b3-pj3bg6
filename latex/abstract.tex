\renewcommand{\baselinestretch}{1.5} %設定行距
\pagenumbering{roman} %設定頁數為羅馬數字
\clearpage  %設定頁數開始編譯
\sectionef
\addcontentsline{toc}{chapter}{摘~~~要} %將摘要加入目錄
\begin{center}
\LARGE\textbf{摘~~要}\\
\end{center}
\begin{flushleft}
\fontsize{14pt}{20pt}\sectionef\hspace{12pt}\quad 由於矩陣計算、自動求導技術、開源開發環境、多核GPU運算硬體等這四大發展趨勢,促使AI領域快速發展,藉由這樣的契機,將實體機電系統透過虛擬化訓練提高訓練效率,再將訓練完的模型應用到實體上。\\[12pt]

\fontsize{14pt}{20pt}\sectionef\hspace{12pt}\quad 此專案是對雙輪車進行設計改良, 以提升行進與對戰效能, 採用 CAD 進行場景與多輪車零組件設計後, 轉入足球場景中以鍵盤 arrow keys 與 wzas 等按鍵進行控制, 對陣雙方每組將有四名輪車球員, 且每兩人在同一台電腦上操作。\\[12pt]

\end{flushleft}
\newpage
%=--------------------Abstract----------------------=%
\renewcommand{\baselinestretch}{1.5} %設定行距
\addcontentsline{toc}{chapter}{Abstract} %將摘要加入目錄
\begin{center}
\LARGE\textbf\sectionef{Abstract}\\
\begin{flushleft}
\fontsize{14pt}{16pt}\sectionef\hspace{12pt}\quad Due to the four major development trends of multidimensional arrays  computing, automatic differentiation, open source development environment, and multi-core GPUs computing hardware. The rapid development of the AI field has been promoted. In view of this development, the physical mechatronic systems can gain machine learning efficiency through their simulated virtual system training process. And afterwards to apply the trained model into real mechatronic systems.\\[12pt]

\fontsize{14pt}{16pt}\sectionef\hspace{12pt}\quad This project focuses on the design improvement of a two-wheeled vehicle, aiming to enhance its performance in movement and combat. CAD is used to design the scenes and components of the multi-wheeled vehicle. Afterward, the vehicle is placed in a soccer field scenario, and control is achieved using keyboard arrow keys and other buttons such as "wzas." Each team will have four players controlling the vehicles, with each pair operating on the same computer.\\
\end{flushleft}
